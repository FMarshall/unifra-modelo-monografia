\begin{resumo}{Resumo. Abstract. Metodologia e resultados. Conclusões. }
\label{sec:resumo}
De acordo com a \cite[NBR6028/2003]{NBR6028:2003}, da ABNT, no texto do RESUMO devem constar algumas
idéias-chave. A primeira frase deve ser significativa, explicando o tema principal. A
apresentação dos objetivos justifica a realização da pesquisa. A apresentação dos recursos
técnicos e metodológicos mostra o caminho da realização da pesquisa. A descrição da
estrutura do texto corresponde ao desenvolvimento. E, por último, a apresentação da síntese
dos resultados (conclusão) confirma os objetivos e o percurso metodológico. Deve ser
digitado em fonte de tamanho 12 pt, em espaço simples, na voz ativa, na terceira pessoa do
singular e é recomendado que contenha de 150 a 250 palavras, para trabalhos acadêmicos e
relatórios técnicos, e sejam também incluídas palavras representativas do conteúdo. Em casos
especiais é permitido um resumo que ultrapasse o número de palavras indicadas, desde que
não ultrapassem, o RESUMO e o ABSTRACT, uma página. Logo abaixo do RESUMO, são
escritas Palavras-chave, em negrito, separadas entre si por ponto e finalizadas também por
ponto. O RESUMO deve ser traduzido para o inglês (ABSTRACT). O termo RESUMO
precede o texto e é título sem indicativo numérico. É adotado o justificado à esquerda,
contrário à NBR 14724/2005, que indica o centralizado. Se necessário, para elaboração de um
resumo com mais palavras, pode ser utilizado o tamanho de fonte 10 pt, mantendo ambos,
RESUMO e ABSTRACT, numa mesma página
\end{resumo}
\begin{abstract}{Resumo. Abstract. Metodology and results. Conclusions.}
\label{sec:abstract}
In agreement with \cite[NBR6028/2003]{NBR6028:2003}, of ABNT, in the ABSTRACT should consist some ideakey, a summary of the work. 
The first sentence should be significant explaining the main
theme. Next, to present the objectives which justify the accomplishment of the research. Also,
describe the technical and methodological resources that was used to accomplishment of the
research and the description of the structure of the text corresponding to the development.
And, last, the presentation of the synthesis of the results (conclusion) reached in function of
the objectives and of the methodological course. All this should be typed in 12 pt font and in
simple space. The ABSTRACT should be written in the voice activates and in the third person
of the singular and is recommended to contain from 150 to 250 words, for academic works
and technical reports and, also, be included, representative words of the content. In special
cases is allowed a summary to exceed the number of suitable words, but that doesn't pass in a
single page, write RESUMO and ABSTRACT in the same page. The Key words, soon below
the ABSTRACT, in bold face, separated amongst themselves by point and also concluded by
point. The ABSTRACT precedes the text and is a title without indicative numeric, it should be
left justified, in opposite to NBR 14724/2005 that indicate centralized. Is possible to use font
10 pt to write a larger ABSTRACT than 250 words, since the RESUMO and ABSTRACT
remain in the same page.
\end{abstract}