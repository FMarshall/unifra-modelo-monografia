%%-%%-%%-%%-%%-%%-%%-%%-%%-%%-%%-%%-%%-%%-%%-%%-%%-%%-%%-%%-%%
%%                                                          %%
%%              tfg.tex - v0.04 Hinkelmann                  %%
%%                                                          %%
%% Unifra - Centro Universitário Franciscano                %%
%% Modelo de Trabalho acadêmico                             %%
%%                                                          %%
%% Guilherme Hinkelmann <guilherme@hinkelmann.com.br>       %%
%%                                                          %%
%%-%%-%%-%%-%%-%%-%%-%%-%%-%%-%%-%%-%%-%%-%%-%%-%%-%%-%%-%%-%% 
\documentclass[monografia]{unifra-abntex2}

% ---
% PACOTES
% ---
%\usepackage{cmap}				% Mapear caracteres especiais no PDF
\usepackage{pslatex}			% Usa a fonte Times News Roman			
\usepackage{makeidx}            % Cria o indice
\usepackage{hyperref}  			% Controla a formação do índice
\usepackage{lastpage}			% Usado pela Ficha catalográfica
\usepackage{indentfirst}		% Indenta o primeiro parágrafo de cada seção.
\usepackage{nomencl} 			% Lista de simbolos
\usepackage{graphicx}			% Inclusão de gráficos
\usepackage{lipsum}				% para geração de dummy text
%\usepackage[printonlyused]{acronym}
%\usepackage[table]{xcolor}

% ----------------------------------------------------------
% Informações de dados para CAPA e FOLHA DE ROSTO
% ----------------------------------------------------------
%
% Título:
%	1. Título em português
%	2. Título em inglês

\titulo{COMPUTAÇÃO QUÂNTICA:\\ SIMULAÇÃO E LINGUAGENS DE PROGRAMAÇÃO QUÂNTICA}{}%

%
% Autor:
%	1. Nome completo do autor
%	2. Formato de nome para bibliografia

\autor{Guilherme Hinkelmann}
      {Hinkelmann, Guilherme}
%
% Orientador
%   1. Título e nome completo do Orientador
%   2. Nome da Instituição abreviado
\orientador{Prof. Dr. Ciclano Master Dinossauro}{UNIFRA}

%
% Local
%   1. Cidade onde será defendia a monográfia 
% Data
%   1.ano da defesa da monografia
\local{Santa Maria, RS}
\data{2013}

% ----
% Informações sobre a Instituição de ensino e sobre o curso
% ----
%
% Intituição
%   1. Nome da instuição
% Área do curso
%   1. Nome da área que curso pertence
% Curso
%   1. Nome do Curso
% Grau do curso
%   1.Nome do grau do curso
\instituicao{Centro Universitário Franciscano}
\cursoarea{Ciências Tecnológicas}
\cursonome{Sistemas de Informação}
\cursograu{Bacharel} 


% ----
% Informações para a Folha de Aprovação
% ----
% Data de Aprovação
\dataaprovacao{ ........ }{ ....................................... }{ ...............}

% Banca
%   1. Titulo e nome completo do professor da banca
%   2. Nome da Instituição abreviado
\bancaa{Prof. Dr. Fulano de Bagual}{UNIFRA}
\bancab{Prof. Dr. Ciclano de Louco}{UNIFRA}
% ----------------------------------------------------------
% compila o indice
% ----------------------------------------------------------
\makeindex
% ----------------------------------------------------------
% Compila a lista de abreviaturas e siglas
% ----------------------------------------------------------
%\makenomenclature
% ----------------------------------------------------------
% Inserir folha de aprovação digitalizada com as assinaturas
% ----------------------------------------------------------
%\inserirfolhaaprovacao{folhaAprovacao.pdf}


% ----------------------------------------------------------
% Início do documento
% ----------------------------------------------------------

\begin{document}


% ----------------------------------------------------------
% ELEMENTOS PRÉ-TEXTUAIS
% ----------------------------------------------------------
\pretextual
% ---
% Insere Capa, Folha de rosto e folha de aprovação (se inserida).
% ---
\maketitle
% ----------------------------------------------------------
% Imprimir Dedicatória, Agradecimentos, Epígrafe
% ----------------------------------------------------------
%\imprimirdedicatoria{}
%\imprimiragradecimentos{ }

\imprimirepigrafe{A cry of war emerges\\
Echoes over the field\\
Boldley charging the enemy lines\\
-Amon Amarth
}
% ---
% RESUMO e ABSTRACT
% ---
\begin{resumo}{Resumo. Abstract. Metodologia e resultados. Conclusões. }
\label{sec:resumo}
De acordo com a \cite[NBR6028/2003]{NBR6028:2003}, da ABNT, no texto do RESUMO devem constar algumas
idéias-chave. A primeira frase deve ser significativa, explicando o tema principal. A
apresentação dos objetivos justifica a realização da pesquisa. A apresentação dos recursos
técnicos e metodológicos mostra o caminho da realização da pesquisa. A descrição da
estrutura do texto corresponde ao desenvolvimento. E, por último, a apresentação da síntese
dos resultados (conclusão) confirma os objetivos e o percurso metodológico. Deve ser
digitado em fonte de tamanho 12 pt, em espaço simples, na voz ativa, na terceira pessoa do
singular e é recomendado que contenha de 150 a 250 palavras, para trabalhos acadêmicos e
relatórios técnicos, e sejam também incluídas palavras representativas do conteúdo. Em casos
especiais é permitido um resumo que ultrapasse o número de palavras indicadas, desde que
não ultrapassem, o RESUMO e o ABSTRACT, uma página. Logo abaixo do RESUMO, são
escritas Palavras-chave, em negrito, separadas entre si por ponto e finalizadas também por
ponto. O RESUMO deve ser traduzido para o inglês (ABSTRACT). O termo RESUMO
precede o texto e é título sem indicativo numérico. É adotado o justificado à esquerda,
contrário à NBR 14724/2005, que indica o centralizado. Se necessário, para elaboração de um
resumo com mais palavras, pode ser utilizado o tamanho de fonte 10 pt, mantendo ambos,
RESUMO e ABSTRACT, numa mesma página
\end{resumo}
\begin{abstract}{Resumo. Abstract. Metodology and results. Conclusions.}
\label{sec:abstract}
In agreement with \cite[NBR6028/2003]{NBR6028:2003}, of ABNT, in the ABSTRACT should consist some ideakey, a summary of the work. 
The first sentence should be significant explaining the main
theme. Next, to present the objectives which justify the accomplishment of the research. Also,
describe the technical and methodological resources that was used to accomplishment of the
research and the description of the structure of the text corresponding to the development.
And, last, the presentation of the synthesis of the results (conclusion) reached in function of
the objectives and of the methodological course. All this should be typed in 12 pt font and in
simple space. The ABSTRACT should be written in the voice activates and in the third person
of the singular and is recommended to contain from 150 to 250 words, for academic works
and technical reports and, also, be included, representative words of the content. In special
cases is allowed a summary to exceed the number of suitable words, but that doesn't pass in a
single page, write RESUMO and ABSTRACT in the same page. The Key words, soon below
the ABSTRACT, in bold face, separated amongst themselves by point and also concluded by
point. The ABSTRACT precedes the text and is a title without indicative numeric, it should be
left justified, in opposite to NBR 14724/2005 that indicate centralized. Is possible to use font
10 pt to write a larger ABSTRACT than 250 words, since the RESUMO and ABSTRACT
remain in the same page.
\end{abstract}
% ----------------------------------------------------------
% inserir lista de ilustrações, tabelas, Abreviaturas e Sumário
% ----------------------------------------------------------
%\listailustracoes
\listatabelas
%\listasiglas{abrev/Abreviaturas}
\sumario

% ----------------------------------------------------------
% ELEMENTOS TEXTUAIS
% ----------------------------------------------------------
\mainmatter
% ----------------------------------------------------------
% Introdução, Desenvolvimento e Conclusão
% ----------------------------------------------------------
\chapter{INTRODUÇÃO}
\label{chap:introducao}

A introdução deve conter a delimitação do tema, o problema, a justificativa e o
objetivo do projeto, que podem vir em subseções separadas ou não.
É muito importante ressaltar que a delimitação do tema requer clareza a respeito do
campo de conhecimento a que pertence o assunto. O problema é o objeto de pesquisa ou de
estudo. Optou-se, neste exemplo, em separar em subseções a justificativa e o(s) objetivo(s).\\
No caso de projeto de pesquisa, que esteja vinculado a um grupo de pesquisa
institucional, neste item é necessário acrescentar a denominação do grupo, que esteja
devidamente certificado pela Unifra, e a denominação da linha de pesquisa a que pertence o
projeto

\section{JUSTIFICATIVA}
\label{sec:justificativa}
Na justificativa mencionam-se a relevância científica do trabalho, a contribuição da
pesquisa e que benefício poderá trazer à comunidade ou à sociedade. Ainda devem estar claros
o motivo da escolha do tema e as possibilidades de realização da pesquisa.
\section{OBJETIVOS}
\label{sec:objetivos}
A definição dos objetivos determina o que se quer atingir com a realização do
trabalho de pesquisa. Objetivo é sinônimo de meta, fim.
Uma sugestão interessante, na redação dos objetivos, é utilizar, no início das
sentenças, o verbo no infinitivo, tais como: esclarecer tal coisa, definir tal assunto, procurar
aquilo, permitir algo, demonstrar alguma coisa, entre outros.
Alguns autores separam os objetivos em objetivo geral e objetivos específicos, mas
não há regra a ser cumprida quanto a isso. Caso se opte em separá-los, tem-se:
\subsection{OBJETIVOS}
\label{subsec:objetivogeral}
O objetivo geral vincula-se à própria significação geral do tema proposto pelo
projeto, ou seja, significa traçar as principais metas que norteiam a pesquisa.
\subsection{Objetivo específico}
\label{subsec:objetivoespecifico}
Descrever aqui o(s) propósito(s) específico(s) para atingir um ponto de vista do tema,
um ângulo a ser pesquisado, permitindo atingir o objetivo geral. Aconselha-se, na redação
desta seção, não ser prolixo.
\chapter{REFERENCIAL TEÓRICO}
\label{chap:referencial_teorico}

O referencial teórico do assunto deve estar diretamente relacionado à pesquisa.
Requer um levantamento bibliográfico cuidadoso, que deve ser organizado numa seqüência lógica e garantir a fonte (autor, obra, data).\\
A seguir são exemplificadas citações com mais de três linhas e com menos de três
linhas.\\

No que concerne à Política Nacional de Saúde do Idoso, conforme \citeonline[p.24]{gordilho2000desafios} , destacam-se:
\begin{citacao}
a promoção do envelhecimento saudável, a manutenção e a melhoria ao máximo
possível da capacidade funcional dos idosos, a prevenção de doenças, a recuperação
da saúde daqueles que adoecem e a reabilitação daqueles que venham a ter a sua
capacidade funcional restringida.
\end{citacao}
Entre os fatores para a efetiva longevidade do ser humano um tem sido apontado, em
estudos, com muita força: a nutrição. “Várias mudanças decorrentes do processo de
envelhecimento podem ser atenuadas com uma alimentação adequada e balanceada nos
aspectos dietético e nutritivo” \cite[p.31]{salgado2002nutriccao}\\

\section{ EXEMPLO DE SEÇÃO (SEÇÃO SECUNDÁRIA) }
Aqui será alguma coisa...

\label{sec:issoeumasubsecao}

alguns exemplos de citação:\\
\cite{berquo1980fatores}\\
\cite{santos1980dinamica}\\
\cite{NBR6023:2002}\\
\cite{NBR14724:2005}\\
\cite{NBR10520:2002}\\
\cite{lessa2004manual}\\
\cite{rey2000planejar}\\
\cite{rajagopalan2003identidade}\\
\cite{flemming1999calculo}\\
\cite{gonccalves2}\\
\cite{salgado2002nutriccao}
\chapter{METODOLOGIA}
\label{chap:metodologia}
A metodologia deve ser escrita com uma linguagem precisa e técnica, seguindo uma
seqüência cronológica. Representa a descrição formal dos métodos e técnicas a serem
utilizados na pesquisa. Devem constar os métodos de abordagem e de procedimentos, os
instrumentos de coleta de dados (questionário, entrevista, formulários, entre outros), a
delimitação do universo da pesquisa, a delimitação e a seleção da amostra e do tempo
previsto, a equipe de pesquisadores e a divisão do trabalho, as formas de tabulação e o
tratamento dos dados, ou seja, a apresentação de tudo aquilo que será utilizado no trabalho de
pesquisa.\\
Em resumo, a metodologia é a explicação minuciosa, detalhada, rigorosa e exata de
toda ação desenvolvida no método (caminho) do trabalho de pesquisa.
Se a pesquisa envolver seres humanos deve atender à Resolução 196/96, do Conselho
Nacional da Saúde, e ter a aprovação do Comitê de Ética em Pesquisa (Cepe) da Unifra ou de
outra instituição.\\
\section{Notas de rodapé }
As notas de rodapé são detalhadas pela NBR 14724:2011 na seção 5.2.1\footnote{Asnotas devem ser digitadas ou datilografadas dentro das margens, ficando
separadas do texto por um espaço simples de entre as linhas e por filete de 5
cm, a partir da margem esquerda. Devem ser alinhadas, a partir da segunda linha
da mesma nota, abaixo da primeira letra da primeira palavra, de forma a destacar
o expoente, sem espaço entre elas e com fonte menor
\citeonline[5.2.1]{NBR14724:2011}.}\footnote{Caso uma série de notas sejam
criadas sequencialmente, o \abnTeX\ instrui o \LaTeX\ para que uma vírgula seja
colocada após cada número do expoente que indica a nota de rodapé no corpo do
texto.}\footnote{Verifique se os números do expoente possuem uma vírgula para
dividi-los no corpo do texto.}. 
\chapter{RESULTADOS}
\label{chap:resultados}


\section{ORÇAMENTO}
São aqueles materiais que têm uma durabilidade prolongada e normalmente são
definidos como bens duráveis, que não são consumidos durante a realização do trabalho de
pesquisa. Podem ser: geladeiras, condicionadores de ar, microondas, computadores,
impressoras. Por exemplo,\index{tabelas} na \autoref{tab-nivinv} estão discriminados os materiais necessários para o
desenvolvimento do projeto.

\begin{table}[htb]
\begin{center}
    
    \caption[ Discriminação dos materiais permanentes do projeto.
]{ Discriminação dos materiais permanentes do projeto}
    \label{tab-nivinv}
    \begin{tabular} {l|c|c|c}
        \hline
        \textbf{ITEM} & \textbf{QUANTIDADE}  & \textbf{PREÇO UNITÁRIO}  & \textbf{CUSTO(R\$)}  \\
        \hline
        Computador & 1  & 2.700,00 & 2.700,00 \\
        \hline
        Impressora Laser HP & 1  & 600,00 & 600,00 \\
        \hline
        Scanner & 1  & 400,00 & 400,00 \\
        \hline
        Mesa para computador & 1  & 300,00 & 300,00 \\
        \hline
        Cadeira para mesa& 1  & 200,00 & 200,00 \\
         \hline
         \multicolumn{2}{l}{\textbf{TOTAL:}}  & &4.200,00 \\
         \hline
    \end{tabular}
\legend{Fonte: \citeonline{van86}}
\end{center}
\end{table}



\chapter{CONCLUS\~AO}
\label{chap:conclusao}
É a parte final do texto, na qual se apresentam conclusões correspondentes aos
objetivos ou hipóteses \cite[p.6]{NBR14724:2005}. Neste item podem também ser sugeridas ações
futuras a respeito do tema tratado.



% ---
% Finaliza a parte no bookmark do PDF, para que se inicie o bookmark na raiz
% ---
\bookmarksetup{startatroot}% 

% ----------------------------------------------------------
% ELEMENTOS PÓS-TEXTUAIS
% ----------------------------------------------------------
\postextual

% ----------------------------------------------------------
% Referências bibliográficas
% ----------------------------------------------------------
\bibliography{bib/tfg_referencias}


% ----------------------------------------------------------
% Glossário
% ----------------------------------------------------------
%\glossary
%
% ----------------------------------------------------------
% Apêndices e Anexos
% ----------------------------------------------------------
%\begin{apendicesenv}
% include(extra/00_apendice)
%\end{apendicesenv}
%\begin{anexosenv}
%\include{extra/01_anexo}
%\end{anexosenv}


%finaliza o documento 
\end{document}


